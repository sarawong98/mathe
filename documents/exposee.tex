\documentclass[german,a4paper,pdftex,12pt]{article}
\usepackage[T1]{fontenc} % utf8 <- produce real utf8 characters
\usepackage[utf8]{inputenc} % utf8 <- accept utf8 input characters
\usepackage[german]{babel}

\usepackage[hscale=0.75,vscale=0.75,vmarginratio={85:100},heightrounded]{geometry} % less margin at bottom

\usepackage{graphicx}
\usepackage{latexsym}
\usepackage{amsmath, amssymb}
\usepackage{color, eurosym} % 6.9.07
\usepackage{float}
\usepackage{hyperref}
\usepackage{xspace} % set a space if not fullstop / end of sentence
\usepackage{times}

% Seitenformatierungsbefehle
\setlength{\textheight}{220mm}
\setlength{\textwidth}{150mm}
\setlength{\topmargin}{1mm}
\setlength{\headheight}{0mm}
\setlength{\headsep}{0mm}
\setlength{\oddsidemargin}{5mm}
\setlength{\parindent}{32mm}
\setlength{\parskip}{0mm}
\linespread{1.1}

\sloppy  % verhindert, dass Wörter über den Rand herausragen


% Makros
\newcommand{\inv}[1]    {\frac{1}{#1}}
\newcommand{\half}      {\frac{1}{2}}
\newcommand{\R}{{\mathbb R}}
\newcommand{\sect}[1] {\overline{#1}}
\newcommand{\eqn}[2] {\begin{equation} \label{#1} #2 \end{equation}}
\newcommand{\eqnn}[1] {\begin{equation*} #1 \end{equation*}}

\title{% \vfill
%\vspace{-2.0cm}
    3D-Puzzle mit Greifarm}
\author{
    Maja Wantke \texttt{mwantke@stud.hs-bremen.de} \and
    Lara Miritz \texttt{lmiritz@stud.hs-bremen.de} \and
    Nikias Scharnke \texttt{nscharnke@stud.hs-bremen.de} \and
    Sara-Ann Wong \texttt{swong@stud.hs-bremen.de} \\
    Angewandte Mathematik für Medieninformatik \\
    Hochschule Bremen}
\date{ \today}
\parindent 0pt
\parskip 1ex

% Document

\begin{document}

    \maketitle

    \begin{abstract}
        In diesem Projekt wird ein 3D-Puzzle-Spiel entwickelt, das durch einen Roboter-Greifarm gelöst werden soll.
        Dabei wird der Greifarm durch Drag \& Drop interaktiv gesteuert.
    \end{abstract}


    \section{Motivation}
    Roboterarmsimulationen sind ein wesentlicher Bestandteil der Robotikforschung und -ausbildung. Sie ermöglichen es,
    komplexe Bewegungsabläufe zu visualisieren und zu verstehen. Unser Projekt soll eine benutzerfreundliche und
    interaktive Anwendung bereitstellen, die es dem Benutzer ermöglicht, die Funktionsweise eines Roboterarms durch
    direkte Interaktion zu erforschen.


    \section{Verwandte Arbeiten}
    Unser Projekt wurde durch verschiedene Arbeiten im Bereich der Roboterarmsimulation inspiriert.
    Die Bachelorarbeit zur "`Steuerung eines 5 DOF Handhabungsroboters in Arbeitsraumkoordinaten"'\cite{Far01} von Julia Dubcova
    erklärt die Grundlagen der Robotersteuerung und inverse Kinematik, die wir für unseren Roboterarm nutzen.
    Eine weitere wichtige Quelle war die Veröffentlichung "`Tactile Robotic Assembly"'\cite{Far02} vom BBSR.
    Diese half uns, die Interaktion des Roboterarms mit Objekten zu verstehen.
    Beide Arbeiten haben uns wertvolle Ideen und Techniken geliefert, die wir in unserer interaktiven Roboterarmsimulation anwenden wollen.
    Zur Umsetzung der Animationen orientieren wir uns an der neuesten Auflage des Werkes "`Computer Animation: Algorithms and Techniques"'\cite{Far03} von Rick Parent,
    die unter anderem die Grundlagen der Animationsprogrammierung vermittelt.
    Es behandelt Grundlagen wie Bewegung, Deformation und physikalische Simulation, um realistische Animationen zu erstellen.


    \section{Gegenstand der Entwicklung}
    Die voraussichtliche Implementierung umfasst:
    \begin{enumerate}
        \item Erstellung einer GUI mit Tkinter, um die Simulation darzustellen.
        \item Implementierung eines Roboterarms mit zwei Segmenten, der durch inverse Kinematik gesteuert wird.
        \item Möglichkeit für den Benutzer, Ziele per Mausinteraktion festzulegen, die der Roboterarm erreichen soll.
        \item Implementierung von Drag-and-Drop-Funktionalität für das Bewegen von Objekten auf der Leinwand.
    \end{enumerate}

    Beispiele:
    \begin{itemize}
        \item Der Benutzer klickt auf die Leinwand, und der Roboterarm bewegt sich zum angegebenen Punkt.
        \item Der Benutzer kann Puzzleteile greifen und an eine Position im Raster verschieben.
    \end{itemize}

    \section{Nutzung von Tools}
    {\bf Programmiersprache:} Python \\
    {\bf GUI-Toolkit:} Tkinter \\
    {\bf Bibliotheken:} NumPy für mathematische Berechnungen, insbesondere die inverse Kinematik


    \section{Evaluationsmethoden und erwartete Ergebnisse}
    Für das Testen unseres Programms könnten wir Unittests verwenden, um jeden Freiheitsgrad zu testen und die zu erwartenden Ergebnisse zu überprüfen.
    So könnte jede Variable einzeln geprüft werden, ob sie sich wie erwartet verhält.
    Um zu testen, ob die Interaktivität funktioniert, wird geprüft, ob der Roboterarm korrekt auf Benutzerinteraktionen reagiert und die Ziele präzise erreicht.
    Zudem kann Feedback von Testbenutzern eingeholt werden, um die Intuitivität und Benutzerfreundlichkeit der GUI zu bewerten.
    Des Weiteren kann überprüft werden, ob die Simulation flüssig läuft und auf verschiedene Bildschirmgrößen skaliert.


    \section{Arbeitspakete}
    Die voraussichtlichen Aufgaben umfassen:
    \begin{itemize}
        \item Erstellung grundlegender Layouts
        \item Implementierung des Roboterarms
        \item Modellierung des Roboterarms
        \item Implementierung der inversen Kinematik
        \item Implementierung der Zielauswahl durch Klicks
        \item Implementierung der Drag-and-Drop-Funktionalität
        \item Durchführung von Funktionstests
        \item Feedback-Integration und Optimierung
    \end{itemize}

    \newpage

    \section{Zeitplan}
    Die Erstellung des Zeitplans basiert auf der Annahme, dass die endgültige Abgabe des Projekts am 09. Juli 2024 erfolgt:
    \begin{enumerate}
        \item Woche: Initiale GUI-Erstellung und Grundstruktur des Projekts
        \item Woche: Implementierung des Roboterarms und der inversen Kinematik
        \item Woche: Hinzufügen der Interaktionsfunktionen und erster Funktionstest
        \item Woche: Feedback sammeln, Tests und Optimierungen, Abschlussbericht und Präsentation
    \end{enumerate}

    \begin{thebibliography}{20} % Breite für die Nummern
        \bibitem{Far01}
        Julia Dubcova: {\it Steuerung eines 5-DOF-Handhabungsroboters in Arbeitsraumkoordinaten} \\
        URL: https://reposit.haw-hamburg.de/handle/20.500.12738/5263 \\
        (Stand: 10.06.2024 14:30 Uhr)
        \bibitem{Far02}
        Prof. Dr.-Ing. Oliver Tessmann et al.: {\it Tactile Robotic Assembly} \\
        URL:
        \raggedright https://www.bbsr.bund.de/BBSR/DE/veroeffentlichungen/bbsr-online/2024/bbsr-online-05-2024-dl.pdf \\
        (Stand: 10.06.2024 14:25 Uhr)
        \bibitem{Far03}
        Rick Parent, {\it Computer Animation: Algorithms and Techniques} \\
        2. Auflage, Morgan Kaufmann, ISBN 9780124158429
    \end{thebibliography}

\end{document}
