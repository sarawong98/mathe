%! Author = sari3
%! Date = 10.06.2024

% Preamble
\documentclass[12pt]{article}

% Packages
\usepackage{amsmath}
\usepackage{listings}
\usepackage{xcolor}
\usepackage{graphicx}


\lstset{
    language=Python,
    basicstyle=\ttfamily\small,
    keywordstyle=\color{blue},
    stringstyle=\color{red},
    commentstyle=\color{green},
    backgroundcolor=\color{lightgray},
    frame=single,
    numbers=left,
    numberstyle=\tiny\color{gray},
    captionpos=b,
    breaklines=true,
    postbreak=\mbox{\textcolor{red}{$\hookrightarrow$}\space}
}



\title{% \vfill
%\vspace{-2.0cm}
    3D-Puzzle mit Greifarm}
\author{
    Maja Wantke \texttt{mwantke@stud.hs-bremen.de} \and
    Lara Miritz \texttt{lmiritz@stud.hs-bremen.de} \and
    Nikias Scharnke \texttt{nscharnke@stud.hs-bremen.de} \and
    Sara-Ann Wong \texttt{swong@stud.hs-bremen.de} \\
    Angewandte Mathematik für Medieninformatik \\
    Hochschule Bremen}
\date{ \today}
\parindent 0pt
\parskip 1ex

% Document
\begin{document}

    \maketitle

    \begin{abstract}
        Diese Dokumentation beschreibt die Entwicklung und Implementierung einer 2D Roboterarm Simulation,
        die Konzepte der Kinematik praxisnah veranschaulicht. Die Simulation ermöglicht es, einen Roboterarm
        interaktiv zu steuern und Puzzleteile zu bewegen. Nutzer können den Roboterarm selbst konfigurieren
        und so die Komplexität der Steuerung kennenlernen.
    \end{abstract}
    \clearpage
    \tableofcontents
    \clearpage


    \section{Motivation}
        Roboterarmsimulationen sind ein wesentlicher Bestandteil der Robotikforschung. Sie ermöglichen es,
        komplexe Bewegungsabläufe zu visualisieren und zu verstehen. Unser Projekt soll eine benutzerfreundliche
        und interaktive Anwendung bereitstellen, die es dem Benutzer ermöglicht, die Funktionsweise eines 
        Roboterarms durch direkte Interaktion zu erforschen. Der Benutzer kann dabei selbst experimentieren 
        und forschen, um eigene Erkenntnisse über die Bewegungssteuerung und Kinematik des Roboterarms zu gewinnen.
        Dies fördert das Verständnis und die Fähigkeit, theoretisches Wissen in die Praxis umzusetzen.

    \section{Verwandte Arbeiten}
        Unser Projekt wurde durch verschiedene Arbeiten im Bereich der Roboterarmsimulation inspiriert.
        Die Bachelorarbeit zur „Steuerung eines 5-DOF Handhabungsroboters in Arbeitsraumkoordinaten“[1] von Julia
        Dubcova erklärt die Grundlagen der Robotersteuerung und der inversen Kinematik, die wir für unseren
        Roboterarm nutzen. Eine weitere wichtige Quelle war die Veröffentlichung „Tactile Robotic Assembly“[2]
        vom BBSR. Diese half uns, die Interaktion des Roboterarms mit Objekten zu verstehen. Beide Arbeiten haben
        uns wertvolle Ideen und Techniken geliefert, die wir in unserer interaktiven Roboterarmsimulation anwenden
        wollen. Ergänzend haben wir uns auch an der Arbeit „Robot Arm Control using Machine Learning“[3] orientiert,
        welche moderne Ansätze zur Steuerung von Roboterarmen mittels maschinellem Lernen beleuchtet. Zur Umsetzung
        der Animationen orientieren wir uns an der neuesten Auflage des Werkes „Computer Animation: Algorithms and
        Techniques“[4] von Rick Parent, die unter anderem die Grundlagen der Animationsprogrammierung vermittelt.
        Es behandelt Grundlagen wie Bewegung, Deformation und physikalische Simulation, um realistische Animationen
        zu erstellen.

    \section{Projektübersicht}
        In diesem Projekt wurde ein 2D-Puzzle-Spiel entwickelt, das durch einen Roboter-Greifarm gelöst werden
        muss. Der Greifarm wird durch Drag & Drop interaktiv gesteuert und verwendet inverse Kinematik, um
        Bewegungen präzise auszuführen. Diese Steuerung ermöglicht es dem Benutzer, die Komplexität der
        Roboterbewegungen zu verstehen und verschiedene Konfigurationen des Arms auszuprobieren.

        Der Benutzer kann Ziele per Mausinteraktion festlegen, die der Roboterarm erreichen soll, und Puzzleteile
        greifen, um sie an eine Position im Raster zu verschieben. Dabei wird eine Kollisionserkennung verwendet,
        die überprüft, ob eine Kollision zwischen einem Puzzleteil und dem Raster auftritt. Ist dies der Fall, so
        rastet das Puzzleteil im Raster ein. Durch diese interaktive und benutzerfreundliche Anwendung erhält der
        Benutzer die Möglichkeit, die Funktionsweise eines Roboterarms praktisch zu erforschen und dabei eigene
        Experimente durchzuführen. Dies fördert ein tieferes Verständnis der Bewegungsabläufe und der Kinematik
        von Roboterarmen und ermöglicht wertvolle Erkenntnisse durch direkte Interaktion.

    \section{Entwicklungsprozess}
        Der Entwicklungsprozess begann mit der Ideenfindung und Planung, bei der verschiedene Konzepte für die
        Entwicklung eines Spiels, das von einem Roboterarm gelöst werden sollte, erörtert wurden. Das Ziel war es,
        eine benutzerfreundliche Anwendung zu schaffen, die es Nutzern ermöglicht, die Funktionsweise eines
        Roboterarms durch direkte Interaktion zu erforschen.

        Am 10. Juni 2024 wurde das Exposé abgegeben. Zu diesem Zeitpunkt wurde eine Grundstruktur der Anwendung
        erstellt, die das Layout und die grundlegenden Funktionen umfasste. Diese Struktur bildete die Grundlage
        für die weitere Entwicklung.

        Die erste Besprechung, die am 26. Juni 2024 stattfand, ermöglichte eine umfassende Überprüfung des
        bisherigen Fortschritts. Dabei wurde entschieden, die Konfiguration der Anwendung zu erweitern, um den
        Nutzern eine detailliertere Untersuchung der Kinematik des Roboterarms zu ermöglichen. Diese Erweiterung
        war nicht von Anfang an geplant, sondern entstand aus der Erkenntnis heraus, den Benutzern mehr
        Flexibilität bei der Erkundung der Armkinematik zu bieten.

        Am 12. Juli 2024 wurde der aktuelle Stand der Entwicklung vorgestellt. In dieser Präsentation lag der
        Schwerpunkt auf der neuen Konfiguration, die es den Nutzern ermöglicht, die Kinematik des Roboterarms
        selbst zu erforschen. Die Demonstration zeigte, wie Benutzer verschiedene Parameter anpassen und die
        Auswirkungen dieser Anpassungen in Echtzeit beobachten konnten. Diese Funktion ermöglichte eine tiefere
        Interaktion mit der Anwendung und stellte einen wesentlichen Fortschritt im Vergleich zur ursprünglichen
        Planung dar.

        Vor der Präsentation wurden umfassende Tests durchgeführt, um die Funktionsfähigkeit der Implementierung
        sicherzustellen. Dazu gehörten Unittests zur Überprüfung der verschiedenen Komponenten der Anwendung,
        einschließlich der Kollisionserkennung und der Drag-and-Drop-Funktionalität. Diese Tests stellten sicher,
        dass alle Funktionen stabil und fehlerfrei arbeiteten.

        Nach der Präsentation am 12. Juli 2024 wurde noch eine neue Variante, den Roboterarm mit 3 Gelenken zu
        steuern, hinzugefügt, sodass die Maus mit dem Endeffektor überlappt und man die volle Kontrolle bei der
        Steuerung hat. Daneben lag der Fokus auf der Dokumentation und dem Abschlussbericht, in dem die
        Implementierung, die Ergebnisse und das Feedback aus der Präsentation zusammengefasst wurden. Die
        abschließende Bewertung reflektierte die erfolgreiche Umsetzung des Projekts und bot Raum für mögliche
        Verbesserungsvorschläge für zukünftige Entwicklungen.

    \section{Erweiterungsmöglichkeiten}
        Das Projekt bietet mehrere Ansatzpunkte für zukünftige Erweiterungen. Eine wesentliche Möglichkeit besteht
        in der Erweiterung der Konfigurationsmöglichkeiten des Roboterarms. Derzeit können Benutzer grundlegende
        Parameter anpassen, um die Kinematik des Arms zu erforschen. Zukünftige Entwicklungen könnten es ermöglichen,
        noch detailliertere Aspekte der Armkonfiguration zu verändern. Dazu könnten erweiterte Optionen zur
        Feinabstimmung der Gelenkwinkel und Geschwindigkeiten gehören sowie zusätzliche Konfigurationsparameter,
        die eine noch präzisere Anpassung und Untersuchung der Bewegungsabläufe ermöglichen.

        Neben den funktionalen Erweiterungen besteht auch die Möglichkeit zur Verbesserung des Designs der
        Benutzeroberfläche. Die aktuelle Benutzeroberfläche bietet eine funktionale Grundstruktur, jedoch besteht
        Potenzial für eine ansprechendere und intuitivere Gestaltung. Eine Überarbeitung könnte visuelle Hilfsmittel,
        wie interaktive Anleitungen und ansprechende Grafiken, beinhalten. Solche Verbesserungen könnten die
        Benutzerfreundlichkeit erhöhen und die Lernkurve für neue Benutzer verringern, indem sie eine klarere
        und intuitivere Navigation durch die Simulation ermöglichen.

    \section{Schlussfolgerung}
        Das entwickelte 2D-Puzzle-Spiel mit Roboterarmsteuerung bietet eine solide Grundlage für die interaktive
        Simulation von Roboterkinematik. Durch die Möglichkeit, verschiedene Konfigurationen des Roboterarms zu 
        testen, können Benutzer ein tieferes Verständnis für die Funktionsweise und Bewegungsabläufe von Roboterarmen
        entwickeln. Die flexible Konfigurationsmöglichkeit und die Drag-and-Drop-Funktionalität haben sich als
        zentrale Merkmale der Anwendung herausgestellt.
        
        Während der Umsetzung des Projekts haben wir wertvolle Erkenntnisse gewonnen. Besonders hervorzuheben
        ist, dass die Implementierung eines Roboterarms mit drei Gelenken deutlich komplexer und fehleranfälliger
        war als ursprünglich erwartet. Die Herausforderungen bei der mathematischen Modellierung und der
        kinematischen Steuerung haben gezeigt, wie anspruchsvoll die Entwicklung eines solchen Systems sein kann.
        Bei der Simulation traten immer wieder Schwierigkeiten auf, insbesondere bei der exakten Positionierung der 
        Endeffektoren. Die präzise Berechnung der Gelenkwinkel und deren Koordination erwiesen sich als komplexe
        Aufgaben, die häufig Anpassungen und Fehlersuche erforderten. Erst durch eine neue Herangehensweise gelang
        die genaue Positionierung der Endeffektors. Trotz der sorgfältigen Planung und Implementierung sind weiterhin 
        Verbesserungen und Erweiterungen möglich, um die Funktionalität und Benutzerfreundlichkeit weiter zu 
        optimieren. Die gesammelten Erfahrungen haben unsere Kenntnisse in der Roboterkinematik und der praktischen
        Anwendung von Simulationstechniken erheblich erweitert.
    
    \section{Implementierung}
        \subsection{Gegenstand der Entwicklung}
            In einer zweidimensionalen Umgebung soll ein Puzzlespiel entstehen, das mit Hilfe eines Roboterarms
            gesteuert wird. Dieser Arm kann per Drag-and-Drop gesteuert werden und so die einzelnen Puzzleteile
            erreichen und aufheben. Die Anzahl der Gelenke und die Länge der einzelnen Armteile sind dabei über
            ein Menü variabel einstellbar. Sobald das Puzzle gelöst ist, kann es mit Hilfe einer Korrekturfunktion
            überprüft werden.

        \subsection{Roboterarm}
            Um den Roboterarm später steuern zu können, muss dieser zunächst in einer eigenen Klasse konfiguriert
            werden. Dabei wird er mit Startwerten versehen und erhält Methoden, die die Positionen des Roboterarms
            bei der Bewegung neu berechnen.

            \subsubsection{Initialisierung}
                Um später beim Starten des Programms einen Roboterarm sehen zu können, muss dieser initialisiert
                werden. Dabei werden die Anzahl der Gelenke auf zwei sowie die Winkel der drei möglichen Gelenke
                auf 0 Grad festgelegt. Die Länge der einzelnen Armteile wurde zuvor als Konstante festgelegt auf
                jeweils 150 eingestellt.

            \subsubsection{Berechnung Winkel}
                Während der Arm bewegt wird, werden dauerhaft die Winkel in den Gelenken neu berechnet, um den
                Endeffektor erreichen zu können. Benötigt werden dafür die Differenzen der x- und y-Koordinaten
                zwischen dem Endeffektor und dem Schultergelenk beziehungsweise dem Ursprung. Mit diesen kann durch
                den Satz des Pythagoras die Distanz bestimmt werden, die zur Berechnung der Winkel notwendig ist.

                \[
                    c = \sqrt{a^2 + b^2}
                \]
                \[
                    c = \sqrt{80^2 + 60^2} = 100
                \]

                \begin{figure}[h]
                    \centering
                    \includegraphics[width = \linewidth]{Bild 1}
                    \caption{Bild 1}
                \end{figure}

                Für die Berechnung bei 1, 2 und 3 Gelenken ist unterschiedlicher Code erforderlich. Dieser wird je
                nach Bedarf durch eine if-else-Anweisung aufgerufen.

                \subsubsubsection{Berechnung für 1 Gelenk}
                    Handelt es sich um nur ein Gelenk, soll der Endeffektor des Arms in Richtung des Zielpunkts
                    bewegt werden. Für die genauen Koordinaten wird die zuvor berechnete Distanz und der Winkel des
                    Armsegments benötigt. Diesen erhält man durch Einsetzen der Koordinaten in die
                    Arcustangens-Funktion.

                    \[
                        \theta=arctan(\frac{B}{A})
                    \]
                    \[
                        \theta=arctan(\frac{60}{80})=0,64 (36,87^\circ)
                    \]

                    \begin{figure}[h]
                        \centering
                        \includegraphics[width = \linewidth]{Bild 2}
                        \caption{Bild 2}
                    \end{figure}

                    Multipliziert man nun die Armlänge mit dem Sinus beziehungsweise dem Kosinus des Winkels, erhält
                    man den Endeffektor.

                    \[
                        x=c * cos(\theta)
                    \]
                    \[
                        y=c * sin(\theta)
                    \]
                    \[
                        x=100 * cos(0,64)=80
                    \]
                    \[
                        y=100 * sin(0,64)=60
                    \]


                \subsubsubsection{Berechnung für 2 Gelenke}
                    Bei der Berechnung der Winkel mit zwei Armsegmenten wird wie bei einem Armsegment auch zuerst
                    die Distanz zwischen dem Endeffektor und dem Ursprung berechnet. Zusätzlich sind die Längen der
                    beiden Armsegmente gegeben, woraus ein Dreieck mit drei bekannten Seitenlängen entsteht.

                    \begin{figure}[h]
                        \centering
                        \includegraphics[width = \linewidth]{Bild 3}
                        \caption{Bild 3}
                    \end{figure}

                    \[
                        a=50
                    \]
                    \[
                        b=40
                    \]
                    \[
                        c=63,25
                    \]

                    Für die Berechnung des Ellenbogenwinkels kann nun der Kosinussatz umgestellt nach dem Winkel
                    angewendet werden.

                    \[
                        cos(\theta)= \frac{a^2+b^2-c^2}{2*a*b}
                    \]
                    \[
                        cos(\theta)= \frac{50^2+40^2-63,25^2}{2*20*40}
                    \]
                    \[
                        \theta= arccos(0,025) = 89,86^\circ
                    \]

                    Nachdem der Ellenbogenwinkel berechnet wurde, kann mit dessen Hilfe auch der Schulterwinkel
                    berechnet werden, indem folgende Formel verwendet wird.

                    \[
                        \alpha = arctan(\frac{y}{x}) - arctan(\frac{b*cos(\theta)}{a+b*sin(\theta)})
                    \]
                    \[
                        \alpha = arctan(\frac{20}{60}) - arctan(\frac{40*cos(89,86^\circ)}{50+40*sin(89,86^\circ)})                    \]
                    \[
                        \alpha = 18,43^\circ - 37,97^\circ = -19,55^\circ
                    \]

                    Nachdem beide Winkel berechnet wurden, kann mit Hilfe der Armlängen der Arm konstruiert werden.
                    Die Positionen des Ellenbogengelenks und des Endeffektors werden durch Einsetzen der Winkel in
                    Sinus- und Kosinusfunktionen multipliziert mit der Armlänge bestimmt.

                    Ellenbogengelenk:

                    \[
                        x1 = a * cos(\theta)
                    \]
                    \[
                        y1 = a * sin(\theta)
                    \]

                    \[
                        x1 = 50 * cos(-19,55^\circ) = 47,12
                    \]
                    \[
                        y1 = 50 * sin(.19,55^\circ) = 16,73
                    \]

                    Endeffektor:

                    \[
                        x2 = x1 + b * cos(\theta + \alpha)
                    \]
                    \[
                        y2 = y1 + b * sin(\theta + \alpha)
                    \]

                    \[
                        x2 = 47,12 + 40 * cos(-19,55^\circ + 89,86^\circ) = 60,6
                    \]
                    \[
                        y2 = 16,73 + 40 * sin(-19,55^\circ + 89,86^\circ) = 54,4
                    \]

                    Diese Punkte werden miteinander verbunden und werden dauerhaft aktualisiert, wodurch der
                    Roboterarm dargestellt wird und sich den Bewegungen anpasst.

                \subsubsubsection{Berechnung für 3 Gelenke}
                    Die Winkel für die 3 Gelenke können auf verschiedene Weisen berechnet werden. Für die erste
                    Implementierung wurden wie zuvor nur Formeln der Trigonometrie verwendet. Dort wird der
                    Roboterarm von der Ellbogen-Position aus gesteuert. Hier hat man sich zunutze gemacht, dass
                    der Ellbogen- und Handgelenk-Winkel derselbe aber gespiegelt sein kann.
                    Erstmal muss sichergegangen werden, dass das Ziel in Reichweite liegt:

                    \[
                        \sqrt{100^2 + 20^2} < 60 + 50 +70
                    \]
                    \[
                        102 < 180
                    \]

                    \begin{figure}[h]
                        \centering
                        \includegraphics[width = \linewidth]{Bild 4}
                        \caption{Bild 4}
                    \end{figure}

                    Kosinussatz:
                    \[
                        elbowangle = arccos(\frac{(100^2 +20^2)-60^2-50^2}{2*60*50}) = 0,77
                    \]

                    \[
                        k1 = 60 +50 * cos(0,77) = 96
                    \]
                    \[
                        k2 = 50*sin(0,77) = 35
                    \]
                    \[
                        shoulderangle = arctan(\frac{20}{100}) - arctan(\frac{35}{96}) = -0,15
                    \]


                    \[
                        wristtarget = [100, 20] - [70 * cos(0,77 - 0,15), 70* sin(0,77 -0,15)] = [43, -21]
                    \]

                    \[
                        wristdx = 43 - 200 = -157
                    \]

                    \[
                        wristdx = -21 - 200 = -221
                    \]

                    \[
                        wristangle1 = arctan(\frac{-221}{.157})= 0,95
                    \]
                    \[
                        writsangle = 0,77 -0,15 -0,95 = -0,33
                    \]

                    Eine zweite Implementierung nutzt die Formeln, die man durch die gegebenen Werte aufstellen kann, um die fehlenden
                    Winkel rauszubekommen. Das geschieht durch ein numerisches Optimierungsverfahren, das darauf abzielt, die Differenzen
                    zwischen den berechneten Werten und den zuvor gegebenen Winkeln zu minimieren.
                    Die beiden Formeln werden aus der folgenden Skizze ersichtlich:

                    \begin{figure}[h]
                        \centering
                        \includegraphics[width = \linewidth]{Bild 5}
                        \caption{Bild 5}
                    \end{figure}

                    \[
                        dx = armlengths[0] * cos(\alpha) + armlengths[1] * cos(\beta) + armlength[2] * cos(\gamma)
                    \]
                    \[
                        dx = armlengths[0] * sin(\alpha) + armlengths[1] * sin(\beta) + armlength[2] * sin(\gamma)
                    \]

                    \begin{figure}[h]
                        \centering
                        \includegraphics[width = \linewidth]{Bild 6}
                        \caption{Bild 6}
                    \end{figure}

                    In dem Beispiel ist die Ausgangsposition vom Roboterarm auf der linken Skizze abgebildet und die
                    Winkel und Armlängen sind:

                    \[
                        \alpha = 0,5*\pi
                    \]
                    \[
                        \beta = 1,5 * \pi
                    \]
                    \[
                        \gamma = 2,5 * \pi
                    \]

                    \[
                        armlengths[0] = 100
                    \]
                    \[
                        armlengths[1] = 105
                    \]
                    \[
                        armlengths[2] = 110
                    \]

                    \[
                        dx = 250
                    \]
                    \[
                        dy = 150
                    \]

                    Diese Werte werden zum Initial Guess, das sind die Winkel, an denen sich der Algorithmus
                    orientiert.

                    Die Formeln sähen mit den gegebenen Werten folgendermaßen aus:

                    \[
                        0 = 100 * cos(\alpha) + 105 * cos(\beta) + 110 * cos(\gamma) -250
                    \]
                    \[
                        0 = 100 * sin(\alpha) + 105 * sin(\beta) + 110 * sin(\gamma) -150
                    \]

                    Nun wird mit den obigen aufgestellten Formeln und den Orientierungswerten der least squares-
                    Algorithmus aufgerufen. In dem Ergebnis sind dann die optimalen Winkel für den Roboterarm.

                    Das Ergebnis der Rechnung lautet somit:
                    \[
                        self.shoulderangle = 0,27 *\pi
                    \]
                    \[
                        self.elbowangle = 1,72 *\pi
                    \]
                    \[
                        self.wristangle = 2,25 *\pi
                    \]

                    Schließlich müssen auch noch die Positionen aller Gelenke und des Endeffektors berechnet werden.
                    Das Verfahren unterscheidet sich gar nicht, zudem wie man in der ersten Variante vorgehen würde.
                    Die Position der Schulter, welche sich beim Bewegen des Roboterarmes nicht mitbewegt, liegt bei
                    [200, 200].

                    \[
                        elbow = [200, 200] + [100 * cos(0,27 * \pi), 100 * sin(0,27*\pi)] = [266, 275]
                    \]
                    \[
                        wrist = [266, 275] + [105 * cos(0,27 * \pi + 1,72 * \pi), 105 * sin(0,27*\pi + 1,72 * \pi)] = [371, 272]
                    \]
                    \[
                        endeffector = [371, 272] + [ 110 * cos(0,27*\pi + 1,72*\pi + 2,25*\pi), 110 * sin(0,27*\pi + 1,72*\pi + 2,25*\pi)] = [451, 347]
                    \]

        \subsection{GUI-Klasse}
            Die GUI-Klasse initialisiert die Hauptkomponenten der Benutzeroberfläche. Sie erstellt ein Canvas-Widget,
            auf dem der Roboterarm und die Puzzleteile angezeigt werden, sowie verschiedene Konfigurationsoptionen für
            die Steuerung des Roboterarms.

            \subsubsection{Konfigurationspanel}
                Das Konfigurationspanel ermöglicht es dem Benutzer, die Anzahl der Gelenke sowie die Längen der Arme
                des Roboterarms anzupassen. Diese Einstellungen werden dabei dynamisch aktualisiert, sodass Änderungen
                in Echtzeit umgesetzt und visuell dargestellt werden.

                Die Methoden “update\_num\_joints()”, “update\_arm\_length1()”,
                ”update\_arm\_length2()” und
                “update\_arm\_length3()” dienen dazu, die Parameter des Roboterarms zu aktualisieren und ihn
                anschließend neu zu zeichnen.
                Die Methode “update\_num\_joints()” aktualisiert die Anpassungsmöglichkeiten des Roboterarms. Je nach
                der gewählten Anzahl der Gelenke werden die entsprechenden Konfigurationsoptionen des Roboterarms
                angezeigt oder ausgeblendet. Nach der Aktualisierung wird der Roboterarm neu gezeichnet, um die
                Änderungen visuell darzustellen.

            \subsubsection{Puzzleteile und Grid (Zielfläche)}
                Die Puzzleteile werden erstellt und zufällig auf dem Canvas platziert. Ein 2x2-Grid dient als
                Zielbereich für die Puzzleteile. Ein Bild ist hinterlegt, aus welchem die Puzzleteile erstellt werden.
                Vorerst bildet ein 2x2-Grid bestehend aus Quadraten das Zielfeld, in welches die Puzzleteile
                hineingelegt werden sollen. Auf dieser Basis wird das Bild auf die benötigte Größe skaliert, sodass
                das Zielfeld vom Bild abgedeckt wird. Anschließend werden die Puzzleteile mit der vorher definierten
                Feldgröße erstellt, indem entsprechend große Teile aus dem Bild ausgeschnitten werden. Jedes
                Puzzleteil wird mit einem Eventlistener verbunden und in eine Liste von Puzzleteilen hinzugefügt.

    
                Das Grid wird auf ähnliche Weise erstellt wie die Puzzleteile. Hierbei werden Quadrate in der zuvor
                definierten Feldgröße ohne Lücken aneinandergesetzt. Abschließend wird jedem Quadrat eine Position
                zugewiesen.
    
            \subsubsection{Puzzleteile bewegen}
                Um den Roboterarm zu bewegen, muss dieser mit einem Linksklick ausgewählt werden. Dabei wird zunächst
                überprüft, ob sich der Abstand zwischen dem Mauszeiger und dem Endeffektor innerhalb eines Radius
                von 10 befindet.

                Während der Bewegung des Arms wird die Methode “on\_drag()” kontinuierlich ausgeführt. In diesem
                Prozess werden fortlaufend die Position des Endeffektors sowie die des ausgewählten Puzzleteils an
                die Position des Mauszeigers angepasst.

            \subsubsection{Puzzleteile aufnehmen und ablegen}
                Durch einen Rechtsklick soll der Endeffektor ein Puzzleteil „greifen“ können. Falls der Greifarm noch
                kein Puzzleteil aufgenommen hat, wird ihm das Puzzleteil zugewiesen, dessen Koordinaten mit der
                Position des Endeffektors übereinstimmen. Danach wird der Endeffektor zentriert auf das Puzzleteil
                positioniert.

                Wenn der Greifarm bereits ein Puzzleteil aufgenommen hat, werden die Koordinaten des Puzzleteils mit 
                den Koordinaten der Quadrate im Zielfeld verglichen. Ein Puzzleteil wird auf einem Quadrat platziert,
                wenn der Abstand zwischen den Koordinaten kleiner als 3 ist. Diese Bedingung wird durch den logischen 
                Ausdruck in der Methode “is\_inside()” überprüft. Ist dies zutreffend, wird das Puzzleteil auf das
                entsprechende Quadrat im Grid gesetzt. Diese Vorgehensweise trägt zur Verbesserung der
                Benutzerfreundlichkeit bei.
    
            \subsubsection{Überprüfung der Position}
                Die Überprüfung beginnt erst nach Betätigung des Knopfes „Check Positions“. Bei diesem Schritt
                werden die hinterlegten Positionen der Puzzleteile mit den aktuellen Positionen verglichen, um
                sicherzustellen, dass alle Teile korrekt platziert sind. Ein Zähler wird nur dann erhöht, wenn die
                Positionen der Puzzleteile mit den vorgegebenen Positionen übereinstimmen. Dieser Zähler gibt nach
                der Überprüfung an, wie viele Puzzleteile korrekt positioniert wurden.

        \subsection{Kollisionserkennung}
            Diese Mechanismen tragen dazu bei, eine präzise und benutzerfreundliche Interaktion mit dem 
            Roboterarm und den Puzzleteilen zu gewährleisten. Die Kollisionserkennung ist entscheidend für
            die korrekte Funktionalität der Anwendung und die Genauigkeit der Interaktionen.
            
            \subsubsection{Kollisionserkennung beim Drag \& Drop von Puzzleteilen}
                Ein zentraler Aspekt der Kollisionserkennung betrifft das Drag & Drop von Puzzleteilen. Hierbei wird
                die Methode “on\_right\_click(self, event)” verwendet, um den Greifarm Puzzleteile mit der rechten
                Maustaste aufnehmen oder ablegen zu lassen. Um sicherzustellen, dass das Puzzleteil korrekt auf
                einem Quadrat positioniert wird, wird die Methode “is\_inside(self, square\_coords, piece\_coords)”
                verwendet. Diese Methode überprüft, ob die Randkoordinaten des Puzzleteils innerhalb der Grenzen 
                eines Grid-Quadrats liegen, wobei eine Toleranz von 3 berücksichtigt wird. Diese Toleranz hilft,
                kleine Abweichungen in der Positionierung auszugleichen und gewährleistet eine präzise Platzierung
                der Puzzleteile, was zu einer höheren Benutzerfreundlichkeit beiträgt.
    
            \subsubsection{Kollisionserkennung bei der Positionierung des Endeffektors}
                Zusätzlich zur Platzierung von Puzzleteilen ist die Kollisionserkennung auch bei der Positionierung
                des Endeffektors von Bedeutung. Die Methode “on\_press(self, event)” wird aktiviert, wenn der Benutzer
                die Maus in der Nähe des Endeffektors klickt, um den Drag-Modus zu starten. Um festzustellen, ob sich
                der Mauszeiger in der Nähe des Endeffektors befindet, verwendet die Methode
                “is\_near\_end\_effector(self, x, y)” die euklidische Distanz zwischen der Position des Mauszeigers und
                der Position des Endeffektors. Ein Radius von 10 wird hierbei als Kriterium verwendet, um die Nähe
                zu überprüfen.

            \subsubsection{Kollisionserkennung während der Armbewegung}
                Während der Armbewegung wird die Methode “on\_drag(self, event)” aufgerufen, die es ermöglicht, dass
                das ausgewählte Puzzleteil der Bewegung des Roboterarms folgt. Hierbei wird sichergestellt, dass das
                Puzzleteil immer entsprechend der Position des Endeffektors aktualisiert wird.

\end{document}